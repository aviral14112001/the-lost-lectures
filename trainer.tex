\documentclass[13pt,oneside,a4paper]{book}
\usepackage[margin=2cm]{geometry}
\usepackage{fontspec}
\usepackage[ddmmyyyy,hhmmss]{datetime}
\usepackage{lmodern}
\usepackage{hyperref}
\usepackage{graphicx}
\usepackage{xcolor}
\usepackage{menukeys}
\usepackage{amsmath}
\hypersetup{
    colorlinks,
    citecolor=red,
    filecolor=blue,
    linkcolor=blue,
    urlcolor=blue
}

\graphicspath{{./assets}}
\setlength{\parindent}{0pt}

\begin{document}
\begin{titlepage}
\begin{center}
        \pagebreak

        \hspace{0pt}
        \vfill
        {\fontsize{32pt}{10pt}\selectfont Group 3 Manual}\\
        \vspace{3cm}
        \textsc{\Large Mentored by} \\

        {\large Anshul, Ujjwal and Manas} \\
        \vspace{1cm}
        \textbf{Compiled~on~\today~at~\currenttime}

        \vspace*{\fill}
\end{center}

        {\fontsize{11pt}{10pt}\selectfont
               This document will be updated regularly so make sure you are viewing the latest copy. 
        }
\end{titlepage}

\pagenumbering{gobble}
\tableofcontents
\clearpage
\pagenumbering{arabic}

\chapter{Assignment 1}
\section{System Setup}
\begin{itemize}
        \item Install an editor/IDE
        \item Choose a Language
        \item Install a compiler
\end{itemize}

\clearpage

\chapter{Switching from C to C++}
\section{Why so serious?}

\begin{verbatim}
#include <stdio.h>

int main() {

        int n;
        long long m;
        char s[1000];

        scanf("%d%lld%s", &n, &m, &s);

        printf("%d %lld %s\n", n, s);

        return 0;
}
\end{verbatim}

becomes

\begin{verbatim}
#include <bits/stdc++.h>
using namespace std;

int main() {

        int n;
        long long m;
        char s[1000];

        \\ a new data structure for strings which makes life easy
        string ss;

        cin >> n >> m >> s >> ss;

        cout << n << ' ' << m << ' ' << s << ' ' << ss << '\n';

        return 0;
}
\end{verbatim}


\clearpage

\chapter{Assigment 2}
\section{How much do you know?}
The aim is to gauge your level of expertise so that we can provide everyone with what they want without wasting
anybody's time. Don't cheat or lie, it will defeat the purpose of the assignment.

\subsection{Level 1}
\begin{enumerate}
        \item \href{https://leetcode.com/problems/fibonacci-number/}{Fabonacci}
        \item \href{https://codeforces.com/contest/4/problem/A}{Watermelon}
\end{enumerate}

\subsection{Level 2}
\begin{enumerate}
        \item \href{https://leetcode.com/problems/move-zeroes/}{Zeroes to the end} If you do it in O(n\textsuperscript{2}) we won't mind, but leetcode might give a TLE.
        \item \href{https://leetcode.com/problems/count-primes/}{Count Primes}
\end{enumerate}

\subsection{Level 3}
\begin{enumerate}
        \item \href{https://leetcode.com/problems/move-zeroes/}{Zeroes to the end} IMP: In a single linear O(n) traversal
        \item \href{https://leetcode.com/problems/convert-to-base-2/}{Convert to base 2} (because that BE course was not a joke)
\end{enumerate}

Adding the entry to the google sheet

If you solved 1 from L1; 1, 2 from L2 and none from L3, you write it as \textbf{``10 11 00''}.\\
Someone who solves all will write \textbf{``11 11 11''} and will color his box green.\\
Partial solvers color their box blue.\\
Those who tried but solved none will color their's red.\\

This will be followed for all the future assignments. This assignment needs to be done within next 4 days, and then we will have a little discussion if required.

Indenting the code is very important, I personally follow the K\&R style but you can choose Google, LLVM or clang. Use your left pinky finger to hit the \keys{\tab} key.

\clearpage
 
\chapter{Baby Steps}
\section{Making I/O fast in C++}
These two lines of code can make your C++ runtimes faster by 50\%, you only notice
the difference on large testcases.

\begin{verbatim}
        ios_base::sync_with_stdio(false);
        cin.tie(NULL), cout.tie(NULL);
\end{verbatim}

\section{A basic template file}

A template is a piece of code you can copy everytime you solve some problem.
\smallskip

If the  problem has no test cases:
\begin{verbatim}
#include <bits/stdc++.h>
using namespace std;

#define fastIO ios_base::sync_with_stdio(0 && cin.tie(0) && cout.tie(0));

typedef long long ll;

int main() {
        fastIO;

        // insert your code here ....
        
        return 0;
}
\end{verbatim}

\hrule
\bigskip

If the problem has test cases:

\begin{verbatim}
#include <bits/stdc++.h>
using namespace std;

#define fastIO ios_base::sync_with_stdio(0 && cin.tie(0) && cout.tie(0));

typedef long long ll;

// this solve() function can be treated exactly like main()
int solve() {
        
        // insert your code here
        
        return 0;
}

int main() {
        fastIO;

        int t;
        cin >> t;
        while(t--) {
                solve();
        }
        
        return 0;
}
\end{verbatim}

\hrule\bigskip

\section{Frequently seen numbers}

There are a few numbers like 10\textsuperscript{7}, 10\textsuperscript{9} + 7 (1000000007) , etc. which you
will frequently find in problems, we will discuss what they mean over here.

\begin{itemize}
        \item You can run 10\textsuperscript{7 to 8} iterations in around one second, so
                mostly you will see that they give size n of array equal to 10\textsuperscript{5} 
                and number of test cases of the order of 10\textsuperscript{3} or 10\textsuperscript{2}.
        \item 10\textsuperscript{9} + 7 (1000000007) is provided as a huge prime number you will need
                to take a modulo with of your answers in combinatorics type problems, where the answer can be so large
                that it will overflow the 64 bit integer (long long) too! You will need to know the modulo
                arithmetic rules for solving such problems. 998244353 is another prime number which might be
                found in a few problems.
        \item 10\textsuperscript{-9}~\leq~x~\leq~10\textsuperscript{9}, means you need to use \textit{int} for storing x, anything
                larger than that goes inside the 64 bit long long, which will mostly be mentioned as x~\leq~10\textsuperscript{18}.
\end{itemize}

\section{Common Terminology}

I have added the ones which always confuse me, will add more as
they come to my mind.

\begin{itemize}
        \item \textbf{substring:} A continuous piece cut from a string. For e.g. ``abcd'' is a substring of ``aaabcdddeee'' but ``abcde'' is not
        \item \textbf{subsequence:} A discontinuous piece of an array or string. In the above example ``abcde'' is a subsequence of ``aaabcdddeee''.
        \item \textbf{subarray:} It is the \textbf{array} version of \textbf{substring}.
        \item \textbf{segment:} It is another name for a subarray.
\end{itemize}

\end{document}
