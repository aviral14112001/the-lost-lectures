\documentclass[13pt, oneside, a4paper]{book}
\usepackage[margin=2cm]{geometry}
\usepackage{fontspec}
\usepackage{lmodern}
\usepackage{hyperref}
\usepackage{graphicx}
\usepackage{xcolor}
\usepackage{menukeys}
\usepackage{amsmath}
\hypersetup{
    colorlinks,
    citecolor=red,
    filecolor=blue,
    linkcolor=blue,
    urlcolor=blue
}

\graphicspath{{./assets}}
\setlength{\parindent}{0pt}

\begin{document}
\begin{titlepage}
\begin{center}
        \pagebreak

        \hspace{0pt}
        \vfill
        {\fontsize{32pt}{10pt}\selectfont Group 3 Manual}\\
        \vspace{3cm}
        \textsc{\Large Mentored by} \\

        {\large Anshul, Ujjwal and Manas} \\
        \vspace{1cm}
        \textbf{Last Updated : \today}

        \vspace*{\fill}
\end{center}

        {\fontsize{11pt}{10pt}\selectfont
               This document will be updated regularly so no need to download it to local. 
        }
\end{titlepage}

\pagenumbering{gobble}
\tableofcontents
\clearpage
\pagenumbering{arabic}

\chapter{Assignment 1}
\section{System Setup}
\begin{itemize}
        \item Install an editor/IDE
        \item Choose a Language
        \item Install a compiler
\end{itemize}

\clearpage

\chapter{Switching from C to C++}
\section{Why so serious?}

\begin{verbatim}
#include <stdio.h>

int main() {

        int n;
        long long m;
        char s[1000];

        scanf("%d%lld%s", &n, &m, &s);

        printf("%d %lld %s\n", n, s);

        return 0;
}
\end{verbatim}

becomes

\begin{verbatim}
#include <bits/stdc++.h>
using namespace std;

int main() {

        int n;
        long long m;
        char s[1000];

        string ss;

        cin >> n >> m >> s >> ss;

        cout << n << ' ' << m << ' ' << s << ' ' << ss << '\n';

        return 0;
}
\end{verbatim}


\clearpage

\chapter{Assigment 2}
\section{How much do you know?}
The aim is to gauge your level of expertise so that we can provide everyone with what they want without wasting
anybody's time. Don't cheat or lie, it will defeat the purpose of the assignment.

\subsection{Level 1}
\begin{enumerate}
        \item \href{https://leetcode.com/problems/fibonacci-number/}{Fabonacci}
        \item \href{https://codeforces.com/contest/4/problem/A}{Watermelon}
\end{enumerate}

\subsection{Level 2}
\begin{enumerate}
        \item \href{https://leetcode.com/problems/move-zeroes/}{Zeroes to the end} If you do it in O(n\textsuperscript{2}) we won't mind.
        \item \href{https://leetcode.com/problems/count-primes/}{Count Primes}
\end{enumerate}

\subsection{Level 3}
\begin{enumerate}
        \item \href{https://leetcode.com/problems/move-zeroes/}{Zeroes to the end} IMP: In a single linear O(n) traversal
        \item \href{https://leetcode.com/problems/convert-to-base-2/}{Convert to base 2} (because that BE course was not a joke)
\end{enumerate}

Adding the entry to the google sheet

If you solved 1 from L1; 1, 2 from L2 and none from L3, you write it as \textbf{``10 11 00''}.\\
Someone who solves all will write \textbf{``11 11 11''} and will color his box green.\\
Partial solvers color their box blue.\\
Those who tried but solved none will color their's red.\\

This will be followed for all the future assignments. This assignment needs to be done within next 4 days, and then we will have a little discussion if required.

Indenting the code is very important, I personally follow the K\&R style but you can choose Google, LLVM or clang. Use your left pinky finger to hit the \keys{\tab} key.

\clearpage
 
\end{document}
